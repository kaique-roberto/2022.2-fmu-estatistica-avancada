\documentclass[10pt]{beamer}

% Template predefined settings
\usetheme[progressbar=frametitle]{metropolis}
\usepackage{appendixnumberbeamer}

\usepackage{booktabs}
\usepackage[scale=2]{ccicons}

\usepackage{pgfplots}
\usepgfplotslibrary{dateplot}

\usepackage{xspace}
\newcommand{\themename}{\textbf{\textsc{metropolis}}\xspace}

% Color Themes
% \usecolortheme{wolverine}
% \usecolortheme{beaver}
% \usecolortheme{dolphin}
% \usecolortheme{seagull}
% \usecolortheme{seahorse}
% \usecolortheme{rose}

% Language and basic staff
\usepackage[brazil]{babel}
\usepackage[utf8]{inputenc}

% Formatting
\usepackage[absolute,overlay]{textpos}
\usepackage{bookmark}
\usepackage{array}
\usepackage{graphicx}
\usepackage{animate}
\usepackage{hyperref}
\usepackage{multicol} % Multicols

% Ident First Line
\renewcommand{\indent}{\hspace*{2em}}

% Colors and Design
\usepackage{xcolor}

% Defining some colors
\definecolor{lightblue}{rgb}{0,0.55,0.5}
\definecolor{azeroblue}{RGB}{204,255,255}
\definecolor{strpink}{rgb}{0.858, 0.188, 0.478}

% Math
\usepackage{amsmath,amssymb,amsfonts}
\usepackage{bbm}
\usepackage{bm}
\usepackage[all,2cell]{xy}

% Defining Theorems
\usepackage{amsthm}

% Numbered Theorems in Beamer
\setbeamertemplate{theorems}[numbered]

% Personalized Theorems
% \theoremstyle{definition}
% \newtheorem{defn}{Definição}[section]
% \newtheorem{teo}[defn]{Teorema} % usa mesmo contador da definição
% \newtheorem{lem}[defn]{Lema}
% \newtheorem{cor}[defn]{Corolário}
% \newtheorem{prop}[defn]{Proposição}
% \newtheorem{exerc}{Exercício}
% \newtheorem{ex}[defn]{Exemplo}
% \newtheorem{rem}[defn]{Observação}
% \newtheorem{fat}[defn]{Fato}

% Personalized Theorems with Colors
\theoremstyle{definition}
\newtheorem{defn}{\textcolor{brown}{Definição}}[section]
\newtheorem{teo}[defn]{\textcolor{purple}{Teorema}} % usa mesmo contador da definição
\newtheorem{lem}[defn]{\textcolor{lightblue}{Lema}}
\newtheorem{cor}[defn]{\textcolor{green}{Corolário}} 
\newtheorem{prop}[defn]{\textcolor{blue}{Proposição}} 
\newtheorem{exerc}{\textcolor{red}{Exercício}}[section]
\newtheorem{ex}[defn]{\textcolor{strpink}{Exemplo}}
\newtheorem{rem}[defn]{\textcolor{orange}{Observação}}
\newtheorem{fat}[defn]{\textcolor{yellow}{Fato}}
\newtheorem{conj}[defn]{\textcolor{red}{Conjectura}}

% Declarations
\DeclareGraphicsExtensions{.pdf,.jpg,.png}
\graphicspath{{./figuras/}} % path for images

% Block Wlements with Background Colors
\metroset{block=fill}

% Insert an "E" for sinalizing that I need to write on blackboard
% \tikz[remember picture, overlay] {\node[anchor=south west, outer sep=0pt] at (current page.south west) {\includegraphics[width=0.05\linewidth]{figuras/inserir-exemplo.PNG}};}

% Tikz Diagrams
\usepackage{tikz}

%%%%%%%%%%%%%%%%%%%%%%%%%%%%%%%%%%%%%%%%%
%%% Title
%%%%%%%%%%%%%%%%%%%%%%%%%%%%%%%%%%%%%%%%%
\title{Estatística Avançada - Aula 00}
\subtitle{Fundamentos: Tipos de Variáveis, Estatística Descritiva, Amostragem}
% \date{\today}
\date{}
\author{Kaique Matias de Andrade Roberto}
\institute{Ciências Atuariais - Ciências Econômicas \\
$ $\\
HECSA - Escola de Negócios \\ $ $ \\ FIAM-FAAM-FMU}
\titlegraphic{\hfill\includegraphics[height=1.5cm]{figuras/fmu.PNG}}

%%%%%%%%%%%%%%%%%%%%%%%%%%%%%%%%%%%%%%%%%
%%% Main
%%%%%%%%%%%%%%%%%%%%%%%%%%%%%%%%%%%%%%%%%
\begin{document}

\maketitle

%%% Contents %%%%%%%%%%%%%%%%%%%%%%%%%%%%
\begin{frame}{Conteúdo}
  \setbeamertemplate{section in toc}[sections numbered]
  \tableofcontents%[hideallsubsections]
\end{frame}

% %%%%%%%%%%%%%%%%%%%%%%%%%%%%%%%%%%%%%%%%%
% \section{Conceitos Importantes de Disciplinas Anteriores}
% %%%%%%%%%%%%%%%%%%%%%%%%%%%%%%%%%%%%%%%%%

% %%%%%%%%%%%%%%%%%%%%%%%%%%%%%%%%%%%%%%%%%
% \begin{frame}{Conceitos Importantes de Disciplinas Anteriores}
% \begin{itemize}
%     \item Conjuntos;
%     \item Exponenciais e Logaritmos;
%     \item Funções;
%     \item Probabilidade.
% \end{itemize}
% \end{frame}

%%%%%%%%%%%%%%%%%%%%%%%%%%%%%%%%%%%%%%%%%
\section{Tipos de Variáveis}
%%%%%%%%%%%%%%%%%%%%%%%%%%%%%%%%%%%%%%%%%

%%%%%%%%%%%%%%%%%%%%%%%%%%%%%%%%%%%%%%%%%
\begin{frame}{Tipos de Variáveis}
\begin{defn}
\vfill\indent Medidas que descrevem diferenças em tipo ou natureza indicando a presença ou ausência de uma característica ou propriedade, são chamadas de \textbf{dados não-métricos ou qualitativos}.
\end{defn}
\end{frame}

%%%%%%%%%%%%%%%%%%%%%%%%%%%%%%%%%%%%%%%%%
\begin{frame}{Tipos de Variáveis}
\indent As variáveis não métricas ou qualitativas representam características de um indivíduo, objeto ou elemento que não podem ser medidas ou quantificadas; as respostas são dadas em categorias.
\end{frame}

%%%%%%%%%%%%%%%%%%%%%%%%%%%%%%%%%%%%%%%%%
\begin{frame}{Tipos de Variáveis}
\begin{ex}
\vfill\indent Imagine que um questionário será aplicado para levantar dados da renda familiar de uma amostra de consumidores, com base em determinadas faixas salariais. 
\end{ex}
\end{frame}

%%%%%%%%%%%%%%%%%%%%%%%%%%%%%%%%%%%%%%%%%
\begin{frame}{Tipos de Variáveis}
\begin{block}{}
\vfill A Tabela abaixo apresenta as categorias das variáveis.
\begin{table}[]
\begin{tabular}{|l|l|l|}
\hline
 Classe & Salários Mínimos (SM) & Renda Familiar (em reais) \\ \hline
 A & Acima de $20$ SM & Acima de $24240$ reais \\ \hline
 B & $10$ a $20$ SM & De $12120$ até $24240$ reais \\ \hline
 C & $4$ a $10$ SM & De $4848$ até $12120$ reais \\ \hline
 D & $2$ a $4$ SM &  De $2424$ até $4848$ reais \\ \hline
 E & Até $2$ SM &  Até $2424$ reais\\ \hline
\end{tabular}
\end{table}
\end{block}
\end{frame}

%%%%%%%%%%%%%%%%%%%%%%%%%%%%%%%%%%%%%%%%%
\begin{frame}{Tipos de Variáveis}
\begin{block}{}
\vfill\indent Observe que ambas as variáveis são qualitativas, já que os dados são representados por faixas.
\end{block}
\end{frame}

%%%%%%%%%%%%%%%%%%%%%%%%%%%%%%%%%%%%%%%%%
\begin{frame}{Tipos de Variáveis}
\begin{block}{}
\vfill Porém, é muito comum a classificação incorreta por parte dos pesquisadores quando a variável apresenta valores numéricos nos dados.
\end{block}
\end{frame}

%%%%%%%%%%%%%%%%%%%%%%%%%%%%%%%%%%%%%%%%%
\begin{frame}{Tipos de Variáveis}
\begin{block}{}
\vfill Nesse caso, é possível apenas o cálculo de frequências, e não de medidas-resumo, como média e desvio-padrão. 
\end{block}
\end{frame}

%%%%%%%%%%%%%%%%%%%%%%%%%%%%%%%%%%%%%%%%%
\begin{frame}{Tipos de Variáveis}
\begin{block}{}
\vfill\indent Uma possível tabela de frequências obtidas para cada faixa de renda é apresentada abaixo.
\begin{table}[]
\begin{tabular}{|l|l|}
\hline
 Frequências & Renda Familiar (em Reais) \\ \hline
 $10\%$ & Acima de $24240$ reais \\ \hline
 $18\%$ & De $12120$ até $24240$ reais \\ \hline
 $24\%$ & De $4848$ até $12120$ reais \\ \hline
 $36\%$ & De $2424$ até $4848$ reais \\ \hline
 $12\%$ & Até $2424$ reais \\ \hline
\end{tabular}
\end{table}
\end{block}
\end{frame}

%%%%%%%%%%%%%%%%%%%%%%%%%%%%%%%%%%%%%%%%%
\begin{frame}{Tipos de Variáveis}
\indent Um erro comum encontrado em trabalhos que utilizam variáveis qualitativas representadas por números é o cálculo da média da amostra, ou de qualquer outra medida-resumo.
\end{frame}

%%%%%%%%%%%%%%%%%%%%%%%%%%%%%%%%%%%%%%%%%
\begin{frame}{Tipos de Variáveis}
\indent O pesquisador calcula, inicialmente, a média dos limites de cada faixa, supondo que esse valor corresponde à média real dos consumidores situados naquela faixa; mas como a distribuição dos dados não é necessariamente linear ou simétrica em torno da média, essa hipótese é muitas vezes violada.
\end{frame}

%%%%%%%%%%%%%%%%%%%%%%%%%%%%%%%%%%%%%%%%%
\begin{frame}{Tipos de Variáveis}
\indent Para que haja condições de se calcular medidas-resumo, como média e desvio-padrão, a variável em estudo deve ser, necessariamente, quantitativa.
\end{frame}

%%%%%%%%%%%%%%%%%%%%%%%%%%%%%%%%%%%%%%%%%
\begin{frame}{Tipos de Variáveis}
\begin{defn}
\vfill\indent  Os \textbf{dados métricos ou quantitativos} são utilizados quando indivíduos diferem em quantia ou grau em relação a um atributo em particular.
\end{defn}
\end{frame}

%%%%%%%%%%%%%%%%%%%%%%%%%%%%%%%%%%%%%%%%%
\begin{frame}{Tipos de Variáveis}
\indent As variáveis métricas ou quantitativas representam características de um indivíduo, objeto ou elemento resultantes de uma contagem (conjunto finito de valores) ou de uma mensuração (conjunto infinito de valores).
\end{frame}

%%%%%%%%%%%%%%%%%%%%%%%%%%%%%%%%%%%%%%%%%
\begin{frame}{Tipos de Variáveis}
\indent Há várias maneiras de representar uma variável métrica, como veremos à seguir.
\end{frame}

%%%%%%%%%%%%%%%%%%%%%%%%%%%%%%%%%%%%%%%%%
\begin{frame}{Tipos de Variáveis}
Temos representações Gráficas:
\begin{itemize}
    \item gráfico de linhas;
    \item dispersão;
    \item histograma;
    \item ramo e-folhas;
    \item boxplot;
\end{itemize}
\end{frame}

%%%%%%%%%%%%%%%%%%%%%%%%%%%%%%%%%%%%%%%%%
\begin{frame}{Tipos de Variáveis}
medidas de posição ou localização: 
\begin{itemize}
    \item média;
    \item mediana;
    \item moda;
    \item quartis;
    \item decis;
    \item percentis;
\end{itemize}
\end{frame}

%%%%%%%%%%%%%%%%%%%%%%%%%%%%%%%%%%%%%%%%%
\begin{frame}{Tipos de Variáveis}
medidas de dispersão ou variabilidade:
\begin{itemize}
    \item amplitude;
    \item desvio-médio;
    \item variância;
    \item desvio-padrão;
    \item erro-padrão;
    \item coeficiente de variação.
\end{itemize}
\end{frame}

%%%%%%%%%%%%%%%%%%%%%%%%%%%%%%%%%%%%%%%%%
\begin{frame}{Tipos de Variáveis}
\indent Estas variáveis podem ser discretas ou contínuas. As variáveis discretas podem assumir um conjunto finito ou enumerável de valores que são provenientes, frequentemente, de uma contagem, por exemplo, o número de filhos ($0,1,2,...$).
\end{frame}

%%%%%%%%%%%%%%%%%%%%%%%%%%%%%%%%%%%%%%%%%
\begin{frame}{Tipos de Variáveis}
\indent Já as variáveis contínuas assumem valores pertencentes a um intervalo de números reais, por exemplo, peso ou renda de um indivíduo.
\end{frame}

%%%%%%%%%%%%%%%%%%%%%%%%%%%%%%%%%%%%%%%%%
\begin{frame}{Tipos de Variáveis}
\begin{ex}
\vfill\indent No banco de dados abaixo, as variáveis Idade, Peso e Altura são quantitativas.
\begin{table}[]
\begin{tabular}{|l|l|l|l|}
\hline
 Nome & Idade (anos) & Peso (kg) & Altura (m) \\ \hline
 Mariana & $48$ & $62$ & $1,60$ \\ \hline
 Luiz & $54$ & $84$ & $1,76$ \\ \hline
 Roberta & $41$ & $56$ & $1,62$ \\ \hline
 Leonardo & $30$ & $82$ & $1,90$ \\ \hline
 Melissa & $28$ & $54$ & $1,68$ \\ \hline
 Sandro & $50$ & $70$ & $1,72$ \\ \hline
\end{tabular}
\end{table}
\end{ex}
\end{frame}

%%%%%%%%%%%%%%%%%%%%%%%%%%%%%%%%%%%%%%%%%
\begin{frame}{Tipos de Variáveis}
\indent As variáveis ainda podem ser classificadas de acordo com o nível ou escala de mensuração.
\end{frame}

%%%%%%%%%%%%%%%%%%%%%%%%%%%%%%%%%%%%%%%%%
\begin{frame}{Tipos de Variáveis}
\indent \textbf{Mensuração} é o processo de atribuir números ou rótulos a objetos, pessoas, estados ou eventos de acordo com as regras específicas para representar quantidades ou qualidades dos atributos.
\end{frame}

%%%%%%%%%%%%%%%%%%%%%%%%%%%%%%%%%%%%%%%%%
\begin{frame}{Tipos de Variáveis}
$$\xymatrix@!=1pc{
& &*+[F]{\mbox{Nominal}}\\ 
*+[F]{\mbox{Variável Qualitativa}}\ar[drr]\ar[urr] & & \\
& & *+[F]{\mbox{Ordinal}}
}$$
\end{frame}

%%%%%%%%%%%%%%%%%%%%%%%%%%%%%%%%%%%%%%%%%
\begin{frame}{Tipos de Variáveis}
$$\xymatrix@!=1pc{
& & *+[F]{\mbox{Intervalar}}\\ 
*+[F]{\mbox{Variável Quantitativa}}\ar[drr]\ar[urr] & & \\
& & *+[F]{\mbox{Razão}}
}$$
\end{frame}

%%%%%%%%%%%%%%%%%%%%%%%%%%%%%%%%%%%%%%%%%
\begin{frame}{Tipos de Variáveis}
\begin{defn}
\vfill\indent A \textbf{escala nominal} classifica as unidades em classes ou categorias em relação à caraterística representada, não estabelecendo qualquer relação de grandeza ou de ordem. É denominada nominal porque as categorias se diferenciam apenas pelo nome.
\end{defn}
\end{frame}

%%%%%%%%%%%%%%%%%%%%%%%%%%%%%%%%%%%%%%%%%
\begin{frame}{Tipos de Variáveis}
\indent Podem ser atribuídos rótulos numéricos às categorias das variáveis, porém, operações aritméticas como adição, subtração, multiplicação e divisão sobre esses números não são admissíveis. A escala nominal permite apenas algumas operações aritméticas mais elementares.
\end{frame}

%%%%%%%%%%%%%%%%%%%%%%%%%%%%%%%%%%%%%%%%%
\begin{frame}{Tipos de Variáveis}
\indent Por exemplo, pode-se contar o número de elementos de cada classe ou ainda aplicar testes de hipóteses referentes à distribuição das unidades da população nas classes.
\end{frame}

%%%%%%%%%%%%%%%%%%%%%%%%%%%%%%%%%%%%%%%%%
\begin{frame}{Tipos de Variáveis}
\indent Desta forma, a maioria das estatísticas usuais, como média e desvio-padrão, não tem sentido para variáveis qualitativas de escala nominal.
\end{frame}

%%%%%%%%%%%%%%%%%%%%%%%%%%%%%%%%%%%%%%%%%
\begin{frame}{Tipos de Variáveis}
\begin{ex}
\vfill\indent Como exemplos de variáveis não métricas em escalas nominais, podemos mencionar profissão, religião, cor, estado civil, localização geográfica ou país de origem.
\end{ex}
\tikz[remember picture, overlay] {\node[anchor=south west, outer sep=0pt] at (current page.south west) {\includegraphics[width=0.05\linewidth]{figuras/inserir-exemplo.PNG}};}
\end{frame}

%%%%%%%%%%%%%%%%%%%%%%%%%%%%%%%%%%%%%%%%%
\begin{frame}{Tipos de Variáveis}
\begin{defn}
\vfill\indent Uma variável não métrica em \textbf{escala ordinal} classifica as unidades em classes ou categorias em relação à característica representada, estabelecendo uma relação de ordem entre as unidades das diferentes categorias.
\end{defn}
\end{frame}

\begin{frame}{Tipos de Variáveis}
\indent  A escala ordinal é uma escala de ordenação, designando uma posição relativa das classes segundo uma direção. Qualquer conjunto de valores pode ser atribuído às categorias das variáveis, desde que a ordem entre elas seja respeitada.
\end{frame}

%%%%%%%%%%%%%%%%%%%%%%%%%%%%%%%%%%%%%%%%%
\begin{frame}{Tipos de Variáveis}
\indent Assim como na escala nominal, operações aritméticas (somas, diferenças, multiplicações e divisões) entre esses valores não fazem sentido. Desse modo, a aplicação das estatísticas descritivas usuais também é limitada para variáveis de natureza nominal. 
\end{frame}

%%%%%%%%%%%%%%%%%%%%%%%%%%%%%%%%%%%%%%%%%

%%%%%%%%%%%%%%%%%%%%%%%%%%%%%%%%%%%%%%%%%
\begin{frame}{Tipos de Variáveis}
\begin{ex}
\vfill\indent Exemplos de variáveis ordinais incluem opinião e escalas de preferência de consumidores, grau de escolaridade, classe social, faixa etária, etc.
\end{ex}
\end{frame}

%%%%%%%%%%%%%%%%%%%%%%%%%%%%%%%%%%%%%%%%%
\begin{frame}{Tipos de Variáveis}
\begin{defn}
\vfill\indent A \textbf{escala intervalar}, além de ordenar as unidades quanto à característica mensurada, possui uma unidade de medida constante. A origem ou o ponto zero dessa escala de medida é arbitrário e não expressa ausência de quantidade.
\end{defn}
\end{frame}

%%%%%%%%%%%%%%%%%%%%%%%%%%%%%%%%%%%%%%%%%
\begin{frame}{Tipos de Variáveis}
\begin{ex}
\vfill\indent Um exemplo clássico de escala intervalar é a temperatura medida em graus Celsius ou Fahrenheit. A escolha do zero é arbitrária e diferenças de temperaturas iguais são determinadas por meio da identificação de volumes guais de expansão no líquido usado no termômetro.
\end{ex}
\end{frame}

%%%%%%%%%%%%%%%%%%%%%%%%%%%%%%%%%%%%%%%%%
\begin{frame}{Tipos de Variáveis}
\indent A maioria das estatísticas descritivas pode ser aplicada para dados de variável com escala intervalar, com exceção de estatísticas baseadas na escala de razão, como o coeficiente de variação.
\end{frame}

%%%%%%%%%%%%%%%%%%%%%%%%%%%%%%%%%%%%%%%%%
\begin{frame}{Tipos de Variáveis}
\begin{defn}
\vfill\indent A \textbf{escala de razão} ordena as unidades em relação à característica mensurada e possui uma unidade de medida constante. Por outro lado, a origem (ou ponto zero) é única e o valor zero expressa ausência de quantidade. Dessa forma, é possível saber se um valor em um intervalo específico da escala múltiplo de outro.
\end{defn}
\end{frame}

%%%%%%%%%%%%%%%%%%%%%%%%%%%%%%%%%%%%%%%%%
\begin{frame}{Tipos de Variáveis}
\indent Razões iguais entre valores da escala correspondem a razões iguais entre unidades mensuradas. Assim, escalas de razão são invariantes sob transformações de proporções positivas.
\end{frame}

%%%%%%%%%%%%%%%%%%%%%%%%%%%%%%%%%%%%%%%%%
\begin{frame}{Tipos de Variáveis}
\indent Por exemplo, se uma unidade tem 1 metro e outra 3 metros, pode-se dizer que a última tem uma altura três vezes superior à da primeira.
\end{frame}

%%%%%%%%%%%%%%%%%%%%%%%%%%%%%%%%%%%%%%%%%
\begin{frame}{Tipos de Variáveis}
\indent Dentre as escalas de medida, a escala de razão é a mais elaborada, pois permite o uso de todas as operações aritméticas. Além disso, todas as estatísticas descritivas podem ser aplicadas para dados de uma variável expressa em escala de razão.
\end{frame}

%%%%%%%%%%%%%%%%%%%%%%%%%%%%%%%%%%%%%%%%%
\begin{frame}{Tipos de Variáveis}
\begin{ex}
\vfill\indent Exemplos de variáveis cujos dados podem estar na escala de razão incluem renda, idade, quantidade produzida de determinado produto e distância percorrida.
\end{ex}
\end{frame}

%%%%%%%%%%%%%%%%%%%%%%%%%%%%%%%%%%%%%%%%%
\section{Tópicos de Estatística Descritiva}
%%%%%%%%%%%%%%%%%%%%%%%%%%%%%%%%%%%%%%%%%

%%%%%%%%%%%%%%%%%%%%%%%%%%%%%%%%%%%%%%%%%
\begin{frame}{Tópicos de Estatística Descritiva}
\indent A \textbf{estatística descritiva} descreve e sintetiza as características principais observadas em um conjunto de dados por meio de tabelas, gráficos e medidas-resumo, permitindo ao pesquisador melhor compreensão do comportamento dos dados.
\end{frame}

%%%%%%%%%%%%%%%%%%%%%%%%%%%%%%%%%%%%%%%%%
\begin{frame}{Tópicos de Estatística Descritiva}
\indent A análise é baseada no conjunto de dados em estudo (amostra), sem tirar quaisquer conclusões ou inferências acerca da população.
\end{frame}

%%%%%%%%%%%%%%%%%%%%%%%%%%%%%%%%%%%%%%%%%
\begin{frame}{Tópicos de Estatística Descritiva}
\indent Antes de iniciarmos o uso da estatística descritiva, é necessário identificarmos o tipo de variável a ser estudada.
\end{frame}

%%%%%%%%%%%%%%%%%%%%%%%%%%%%%%%%%%%%%%%%%
\begin{frame}{Tópicos de Estatística Descritiva}
\indent O tipo de variável é crucial no cálculo de estatísticas descritivas e na representação gráfica de resultados.
\end{frame}

%%%%%%%%%%%%%%%%%%%%%%%%%%%%%%%%%%%%%%%%%
\begin{frame}{Tópicos de Estatística Descritiva}
 \begin{center}
  \includegraphics[width=1\linewidth]{figuras/diagram-01.jpg}
 \end{center}
\end{frame}

%%%%%%%%%%%%%%%%%%%%%%%%%%%%%%%%%%%%%%%%%
\begin{frame}{Tópicos de Estatística Descritiva}
\begin{exerc}
\vfill\indent Considerando os dados da Tabela abaixo
\begin{table}[]
\begin{tabular}{|l|l|l|l|}
\hline
 Nome & Idade (anos) & Peso (kg) & Altura (m) \\ \hline
 Mariana & $48$ & $62$ & $1,60$ \\ \hline
 Luiz & $54$ & $84$ & $1,76$ \\ \hline
 Roberta & $41$ & $56$ & $1,62$ \\ \hline
 Leonardo & $30$ & $82$ & $1,90$ \\ \hline
 Melissa & $28$ & $54$ & $1,68$ \\ \hline
 Sandro & $50$ & $70$ & $1,72$ \\ \hline
\end{tabular}
\end{table}
calcule a média, variância e desvio-padrão das variáveis Idade, Peso e Altura.
\end{exerc}
\end{frame}

%%%%%%%%%%%%%%%%%%%%%%%%%%%%%%%%%%%%%%%%%
\begin{frame}{Tópicos de Estatística Descritiva}
\indent Agora vamos realizar (uma parte) da análise descritiva dos dados na planilha motocicletas.xlsx.
\tikz[remember picture, overlay] {\node[anchor=south west, outer sep=0pt] at (current page.south west) {\includegraphics[width=0.05\linewidth]{figuras/inserir-exemplo.PNG}};}
\end{frame}

%%%%%%%%%%%%%%%%%%%%%%%%%%%%%%%%%%%%%%%%%
\section{Introdução à Amostragem}
%%%%%%%%%%%%%%%%%%%%%%%%%%%%%%%%%%%%%%%%%

%%%%%%%%%%%%%%%%%%%%%%%%%%%%%%%%%%%%%%%%%
\begin{frame}{Introdução à Amostragem}
\begin{defn}
\vfill\indent A \textbf{população} é o conjunto com todos os indivíduos, objetos ou elementos a serem estudados, que apresentam uma ou mais características em comum. O \textbf{censo} é o estudo dos dados relativos a todos os elementos da população.
\end{defn}
\end{frame}

%%%%%%%%%%%%%%%%%%%%%%%%%%%%%%%%%%%%%%%%%
\begin{frame}{Introdução à Amostragem}
\indent As populações podem ser finitas ou infinitas. As populações finitas são de tamanho limitado, permitindo que seus elementos sejam contados; já as populações infinitas são de tamanho ilimitado, não permitindo a contagem dos elementos.
\end{frame}

%%%%%%%%%%%%%%%%%%%%%%%%%%%%%%%%%%%%%%%%%
\begin{frame}{Introdução à Amostragem}
\begin{ex}
\vfill\indent Como exemplos de populações finitas, podemos mencionar a quantidade de empregados em determinada empresa, de associados em um clube, de produtos fabricados em determinado período, etc. 
\end{ex}
\end{frame}

%%%%%%%%%%%%%%%%%%%%%%%%%%%%%%%%%%%%%%%%%
\begin{frame}{Introdução à Amostragem}
\indent Quando o número de elementos da população, embora possa ser contado, for muito grande, assumimos que a população é infinita.
\end{frame}

%%%%%%%%%%%%%%%%%%%%%%%%%%%%%%%%%%%%%%%%%
\begin{frame}{Introdução à Amostragem}
\begin{ex}
\vfill\indent Quando o número de
elementos da população, embora possa ser contado, for muito grande, assumimos que a população é infinita. São exemplos de populações consideradas infinitas a quantidade de habitantes no mundo, de residências existentes no Rio de Janeiro, de pontos em uma reta, etc.
\end{ex}
\end{frame}

%%%%%%%%%%%%%%%%%%%%%%%%%%%%%%%%%%%%%%%%%
\begin{frame}{Introdução à Amostragem}
\indent Desta forma, existem situações em que o estudo com todos os elementos da população é impossível ou indesejável, de modo que a alternativa seja extrair um subconjunto da população em análise, denominado \textbf{amostra}.
\end{frame}

%%%%%%%%%%%%%%%%%%%%%%%%%%%%%%%%%%%%%%%%%
\begin{frame}{Introdução à Amostragem}
\indent A amostra deve ser representativa da população em estudo, daí a importância deste capítulo. A partir das informações colhidas na amostra e utilizando procedimentos estatísticos apropriados, os resultados obtidos podem ser utilizados para generalizar, inferir ou tirar conclusões acerca da população (inferência estatística).
\end{frame}

%%%%%%%%%%%%%%%%%%%%%%%%%%%%%%%%%%%%%%%%%
\begin{frame}{Introdução à Amostragem}
 \begin{center}
  \includegraphics[width=0.94\linewidth]{figuras/tecnicas-amostragem.PNG}
 \end{center}
\end{frame}

%%%%%%%%%%%%%%%%%%%%%%%%%%%%%%%%%%%%%%%%%
\begin{frame}{Introdução à Amostragem}
\indent 
\end{frame}

% %%%%%%%%%%%%%%%%%%%%%%%%%%%%%%%%%%%%%%%%%
% \begin{frame}{Introdução à Amostragem}
% \indent 
% \end{frame}


%%%%%%%%%%%%%%%%%%%%%%%%%%%%%%%%%%%%%%%%%
\section{Comentários Finais}
%%%%%%%%%%%%%%%%%%%%%%%%%%%%%%%%%%%%%%%%%

%%%%%%%%%%%%%%%%%%%%%%%%%%%%%%%%%%%%%%%%%
\begin{frame}{Comentários Finais}
\indent Em resumo, na aula de hoje nós:
\begin{itemize}
    \item recapitulamos os tipos de variáveis;
    \item aprendemos como classificar variáveis;
    \item recapitulamos alguns tópicos de Estatística Descritiva;
    \item tivemos um primeiro contato com os conceitos de Amostragem.
\end{itemize}
\end{frame}

%%%%%%%%%%%%%%%%%%%%%%%%%%%%%%%%%%%%%%%%%
\begin{frame}{Comentários Finais}
\indent Nas próximas aulas nós vamos focar em:
\begin{itemize}
    \item variáveis aleatórias discretas e contínuas;
    \item propriedades e exemplos envolvendo tais variáveis.
\end{itemize}
\end{frame}

%%%%%%%%%%%%%%%%%%%%%%%%%%%%%%%%%%%%%%%%%
\begin{frame}{Comentários Finais}
\begin{alertblock}{ATIVIDADE PARA ENTREGAR (E COMPOR A NOTA N1)}
        \vfill\indent Resolva em grupos de até 4 integrantes os Exercícios 0.4-0.6.
\end{alertblock}
\end{frame}

%%%%%%%%%%%%%%%%%%%%%%%%%%%%%%%%%%%%%%%%%
\section{Referências}
%%%%%%%%%%%%%%%%%%%%%%%%%%%%%%%%%%%%%%%%%

%%%%%%%%%%%%%%%%%%%%%%%%%%%%%%%%%%%%%%%%%
\begin{frame}{Referências}
 \begin{center}
  \includegraphics[width=0.4\linewidth]{figuras/excel-spss.jpg}
 \end{center}
\end{frame}

%%%%%%%%%%%%%%%%%%%%%%%%%%%%%%%%%%%%%%%%%
\begin{frame}{Bons Estudos!}
 \begin{center}
  \includegraphics[width=0.79\linewidth]{figuras/cafe.jpg}
 \end{center}
\end{frame}

% %%%%%%%%%%%%%%%%%%%%%%%%%%%%%%%%%%%%%%%%%
% \begin{frame}[allowframebreaks]{References}

%   \bibliography{demo}
%   \bibliographystyle{abbrv}

% \end{frame}

\end{document}